% Metadados da Tese %
\newcommand{\titulo}{Título do seu trabalho }
\newcommand{\autor}{Nome do doutorando}
\newcommand{\instituicao}{Universidade Federal de Minas Gerais}
\newcommand{\centro}{Escola de Música}
\newcommand{\programa}{Programa de Pós-Graduação em Música}
\newcommand{\linhadepesquisa}{Linha de Pesquisa}

\newcommand{\orientador}{Orientador}
\newcommand{\coorientador}{Coorientador}


 

% Banca %
\newcommand{\membroInterno}{Membro interno da banca}
\newcommand{\membroExterno}{Membro externo da banca 1}
\newcommand{\membroExternodois}{Membro externo da banca 2}

% Instituição do Membro Externo
\newcommand{\instituicaoInterno}{Filiação membro interno }
\newcommand{\instituicaoExterno}{Filiação membro externo 1}
\newcommand{\instituicaoExternodois}{Filiação membro externo 2}

% Cidade onde ocorreu a defesa
\newcommand{\cidade}{Belo Horizonte}
% Estado onde Aconteceu a defesa
\newcommand{\estado}{MG}
% Mês da defesa
\newcommand{\mes}{Mês da defesa}
% Dia da defesa
\newcommand{\dia}{Dia da defesa}
% Ano da defesa
\newcommand{\ano}{Ano da defesa}


% Definindo os metadados do PDF
\hypersetup{
    pdftitle={\titulo},
    pdfauthor={\autor},
    pdfsubject={Tese de Doutorado em Música. Univesidade Federal de Minas Gerais},
    pdfkeywords={Palavra chave, Palavra chave, Palavra chave}}
