\setlength{\parskip}{1.5pt}
\chapter[Resultados]{Resultados}


Apresente aqui os seus resultados. Inclua imagens, tabelas, glossário como \gls{midi}, referências \cite{article}.



Exemplo de imagem: 
\begin{figure}[h]
    \centering
\includegraphics[width=0.2\textwidth]{arquivos/pre-texto/Logo_UFMG.png}
    \caption{Legenda da Imagem 3}
    \label{fig3}
\end{figure}

Enumere listas assim:
\begin{enumerate}
    \item Item 1
    \item Item 2
\end{enumerate}

Tabela mais larga:
\begin{table}[h!]
    \centering
    \begin{tabularx}{\textwidth}{|l|X|X|X|}
        \hline
        \textbf{Parâmetro} & \textbf{Definição 1} & \textbf{Definição 2} & \textbf{Definição 3} \\ \hline
        \textbf{Título 1}  & Conteúdo & Conteúdo & Conteúdo \\ \hline
        \textbf{Título 2}  & Conteúdo & Conteúdo & Conteúdo \\ \hline
        \textbf{Título 3}  & Conteúdo & Conteúdo & Conteúdo \\ \hline
        
    \end{tabularx}
    \caption{Legenda da tabela 3.}
    \label{tab:nometabela3}
\end{table}


\section{Adicionando pdf}
Talvez você possa incluir um PDF de algum artigo que foi resultado da pesquisa. Para isso, utilize essa lógica neste ponto do documento\footnote{Adicione o seu pdf e mude a localização do arquivo.}:

\begin{listing}[h!]
    \centering
    \begin{minted}[frame=lines, style=friendly, linenos=true]{latex}
\includepdf[pages=-]{arquivos/artigo.pdf}
    \end{minted}
    \caption{Exemplo \LaTeX}
    \label{lst:codigo-latex}
\end{listing}


Finalizando este arquivo, o próximo é Discussão.tex.

